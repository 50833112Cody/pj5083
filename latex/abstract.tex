\renewcommand{\baselinestretch}{1.5} %設定行距
\pagenumbering{roman} %設定頁數為羅馬數字
\clearpage  %設定頁數開始編譯
\sectionef
\addcontentsline{toc}{chapter}{摘~~~要} %將摘要加入目錄
\begin{center}
\LARGE\textbf{摘~~要}\\
\end{center}
\begin{flushleft}
%\fontsize{14pt}{20pt}\sectionef\hspace{12pt}\quad 由於矩陣計算、自動求導技術、開源開發環境、多核GPU運算硬體等這四大發展趨勢,促使AI領域快速發展,藉由這樣的契機,將實體機電系統透過虛擬化訓練提高訓練效率,再將訓練完的模型應用到實體上。\\[12pt]

\fontsize{14pt}{20pt}\sectionef\hspace{12pt}\quad 此專題介紹了網絡物理桌上足球系統的設計和實施。 該系統將物理組件(例如傳感器和執行器)與計算元素集成在一起,以創建互動且引人入勝的桌上足球體驗。 詳細描述了系統的硬件和軟件組件,並討論了集成這些組件的過程。

除了物理測試外,還使用 CoppeliaSim 仿真評估系統的性能。 CoppeliaSim 是一款機器人仿真軟件,可讓工程師在虛擬環境中對複雜系統進行建模和測試。 在這個項目中,CoppeliaSim 被用來模擬網絡物理桌上足球系統的行為,並確定需要改進的地方。

用戶測試和反饋用於評估系統的性能並確定未來需要改進的領域。 結果證明了網絡物理系統在增強傳統遊戲並為互動和參與提供新機會方面的潛力。\\[12pt]

\end{flushleft}
\begin{center}
\fontsize{14pt}{20pt}\selectfont 關鍵字: 類神經網路、強化學習、\sectionef CoppeliaSim、OpenAI Gym
\end{center}
\newpage
%=--------------------Abstract----------------------=%
\renewcommand{\baselinestretch}{1.5} %設定行距
\addcontentsline{toc}{chapter}{Abstract} %將摘要加入目錄
\begin{center}
\LARGE\textbf\sectionef{Abstract}\\
\begin{flushleft}
\fontsize{14pt}{16pt}\sectionef\hspace{12pt}\This project presents the design and implementation of a cyber-physical foosball system. The system integrates physical components, such as sensors and actuators, with computational elements to create an interactive and engaging foosball experience. The hardware and software components of the system are described in detail, and the process of integrating these components is discussed.

In addition to physical testing, the performance of the system is also evaluated using CoppeliaSim simulation. CoppeliaSim is a robotics simulation software that allows engineers to model and test complex systems in a virtual environment. In this project, CoppeliaSim is used to simulate the behavior of the cyber-physical foosball system and to identify areas for improvement.

User testing and feedback are used to evaluate the performance of the system and to identify areas for future improvement. The results demonstrate the potential of cyber-physical systems to enhance traditional games and provide new opportunities for interaction and engagement.\\[12pt]

%\fontsize{14pt}{16pt}\sectionef\hspace{12pt}\quad This project is to use the physical air hockey to play machine, introduce it into the CoppeliaSim simulation environment and give the corresponding settings, simplify its electromechanical system and use Open AI Gym for training, find an algorithm suitable for this system, and then perform it in the CoppeliaSim simulation environment Feasibility of testing algorithm in practical application. And try to stream CoppeliaSim images to web pages for users to watch or manipulate by setting up a server.\\
\end{flushleft}
\begin{center}
\fontsize{14pt}{16pt}\selectfont\sectionef Keyword:  nerual network、reinforcement learning、 CoppeliaSim、OpenAI Gym
\end{center}
